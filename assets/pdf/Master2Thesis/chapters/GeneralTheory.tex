\section{General Theory}

This chapter is dedicated to the analytical study of general forms of RD-ODE systems. For that, we start by introducing the two definitions

\begin{definition}[Sobolev spaces]
	Let $\Omega \subset \R^d$ be a bounded domain with $\del \Omega \in \mathcal C^1$. We denote by $W^{k, p}(\Omega)$ the Sobolev space defined by
	$$W^{k, p}(\Omega) := \lc u \in L^p(\Omega) : \forall \alpha \in \N^d, \varphi \in \mathcal C^\infty(\Omega), \exists v \in L^p(\Omega) : \int_\Omega v \varphi = (-1)^{|\alpha|} \int_\Omega u \del^\alpha \varphi \rc$$   
\end{definition}

\begin{definition}[Laplace operator on a bounded domain]
	
	We define the \textbf{Laplace operator}  $\Delta$ on a bounded domain $\Omega \subset \R^d$ endowed with Neumann boundary conditions ($\del_{\nu} = 0$ on $\del\Omega$) through the relation
	$$\Delta_\nu = \sum_{i=1}^d \frac{\del^2}{\del x_i^2}.$$
	Furthermore, $-\Delta_\nu$ has a discrete spectrum consisting of a sequence of non-decreasing, non-negative eigenvalues 
	$\{\mu_j\}_{j\ge 0}$ with $\mu_0 = 0$, $\mu_n \xrightarrow{n \to \infty} \infty$ and has domain $\mathrm{Dom}(-\Delta_\nu) = W_\nu^{2,p}(\Omega)$ (the space of functions $u\in W^{2,p}(\Omega)$ such that $\del_\nu u_{\vert \del\Omega} = 0$).
\end{definition}

For the sake of notation, we also  drop the $\nu$ symbol in the Laplace operator. i.e., we will write $\Delta_\nu = \Delta$. 

\subsection{Deriving reaction-diffusion equations}

In this section we seek to derive reaction-diffusion equations from scratch based on famous physical principles. We assume to have enough regularity for each quantity in order to carry out the computations. This derivation has for purpose to give the reader a physical intuition behind such equations. To this end, we first introduce the divergence theorem

\begin{theorem}[Divergence theorem]
	Consider a bounded domain $\Omega \subset \R^d$, enclosed by the surface $\Gamma := \del \Omega \subset \R^{d-1}$. Let $\bm F$ be an arbitrary smooth vector field and the application $\nu$ such that for all $\xi \in S$, $\nu(\xi)$ is the outward-pointing unit vector of $S$ at $\xi$. Finally, if $dA$ denotes the measure $d x_1 ... d x_d$ and $ds$ is the surface measure, then the following equality holds
	
	$$\oint_\Gamma \lp \bm F \cdot \nu \rp ds = \int_\Omega \grad \cdot \lp \bm F\rp d A.$$
\end{theorem}

\begin{proof}
	Various well-known proofs exist in the literature (see \cite{https://doi.org/10.48550/arxiv.0807.0088} for the original derivation by G. Green)
\end{proof}

We are all set, let us proceed. General laws of conservation tell us that if $u$ denotes a physical quantity (heat, amount of cells, gene expression, ...), then

$$\underbrace{\frac{\del}{\del t} \int_{\Omega} u d\Omega}_{\text{amount of material in } \Omega} =  \underbrace{D \oint_\Gamma \lp \grad u \cdot \nu \rp d\Gamma}_{\text{flux of material crossing } \Gamma} + \underbrace{\int_\Omega f(u) d\Omega}_{\text{material created inside} \Omega}.$$

Where we obtained the diffusive flux of material crossing $\Gamma$ using Fick's law. Now using the divergence theorem on the second integral yields

$$\int_{\Omega} \frac{\del u}{\del t} d\Omega =  \int_\Omega D \grad \cdot \lp \grad u \rp d\Omega + \int_\Omega f(u) d\Omega.$$

After reordering the terms, we put everyone under the same integral and find

$$\int_\Omega \lp \frac{\del u}{\del t}(x, t) - D \Delta u (x, t) - f(u(x, t)) \rp d\Omega = 0.$$

Each term being integrated on an arbitrary $\Omega$ with size $\mu(\Omega) > 0$, we deduce that what is under the integral must vanish. i.e., 

\begin{equation}\label{eq:rdeq} \frac{\del u }{\del t}(x, t) = D \Delta u(x, t) + f(x, t)\end{equation}

\subsection{Reaction-Diffusion-ODE problems}

Reaction-diffusion-ODE systems are what one obtains when writing evolution equations for coupled diffusive and non-diffusive quantities. Mathematically, this rewrites under the formed of Until recently, the matrix $D$ in (\ref{eq:rdeq}) was mainly chosen as a diagonal matrix $D = diag(d_1, ..., d_d)$, with $d_i > 0$, $i = 1, ..., d$. Taking a coefficient $d_i = 0$ results in canceling the effect of the Laplace operator, thus transforming equation $i$ into a classical ODE coupled to the PDE system. In this paragraph, we inspect the properties of such systems coupling $k >0$ ODE to $m > 0$ reaction-diffusion equations. For that, let $n \in \N$ such that $m+k = n$ and a bounded domain $\Omega \subset \R^{n}$. Let us consider the $\mathcal{C}^2$-nonlinearities $\bm f : \R^k \times \R^m \longrightarrow \R^{m+k}$ describing the reaction kinetics between each components of the system. We aim to investigate properties of
 
\vspace{1em}
\begin{center}
\begin{tabular}{lcl}
	$\del_t\uu  = \bm{\mathrm D} \Delta \uu + \bm{f}(\uu)$ & & on $\Omega \times \mathbb R^{+}$ \\[0.5em] $\del_\nu u_i = 0$ & &on $\del\Omega \times \mathbb R^{+}$, for $i \in [\![ m+1, k ]\!]$ \\[0.5em]
	$\bm{u}(\cdot, 0) = \bm{u}_0(\cdot)$ & &on $\Omega$.
\end{tabular}
\end{center}


\subsubsection{Existence of solutions}

The goal of this section is to show that for an autonomous abstract Cauchy problem, if the operator $A$ is the infinitesimal generator of an analytic $C_0$-semigroup of operators, then the problem admits a unique solution whose regularity depends on the regularity of the initial condition.

\begin{definition}[Abstract Cauchy problem] La définition du problème de Cauchy abstrait nous vient de E. Hille 
	\begin{align}
		\del_t u = Au + f(u) \notag \\
		u(0) = u_0 \notag
	\end{align}
\end{definition}

\begin{itemize}
	\item Abstract Cauchy Problem
	\item 
\end{itemize}


\begin{theorem}[Spectral Mapping Theorem] 
	Consider a locally compact space $X$. Let $A$ be the generator of a positive, bounded, compact, strongly continuous group on $\mathcal C_0(X)$, denoted by $(L(t))_{t\in\R}.$ Then
	$$\sigma(L(t)) = \overline{\exp(t \sigma(A))}$$
\end{theorem}

\begin{proof}
	see Theorem 1.1. from \cite{Arendt1984}
\end{proof}


%
%\begin{proposition}[Generated Semigroups]
%	The operator $L$ generates an analytic $\mathcal C^0$-semigroup $(T(t))_{t\ge 0}$ through the map $t \longmapsto \exp(tL)$. The system $\del_t \xi = L\xi$ is therefore linearly stable if the growth exponent $w < 0$. It is linearly unstable otherwise.
%\end{proposition}
%
%\begin{definition}[Mild solutions]
%	Let $X$ be a Banach space and take $L$, the infinitesimal generator of a $\mathcal C^0$-semigroup $(T(t))_{t\ge 0}$. Let $\xi : I \times \Omega \longrightarrow \R^{m+k}$ such that $\xi(t, \cdot) \in X$. The function $\xi$ is said to be a mild solution of the system if it obeys
%	$$\xi(t, \cdot) = T(t)\xi(0, \cdot) + \int_{0}^{t} T(t- \tau) \xi(\tau, \cdot) d\tau$$
%\end{definition}


\begin{proof}
	It is a straightforward computation. The result is obtained by expanding and refactoring the expression $\det(A - \lambda)$
\end{proof}

\begin{itemize}
    \item Linear / NonLinear decomposition
    \item Bound on the Nonlinear part $\mathcal N$
    
    \item Why is the spectrum so important here
    \item Define DDI
    \item Maybe include Finn's result on the $d_1$ $d_2$ ? (and add full acknowledgement to his work)
    \item linearization of a PDE system
    \item Solving for the eigenvalue problem   
\end{itemize}