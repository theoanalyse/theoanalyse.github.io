\subsection{Finding equilibria and bounds on parameters}

We recall that a \textit{steady state} of the system is a solution to the PDE, satisfying boundary conditions that does not depend on time, resulting in a vanishing time-derivative in equations. E.g., calling $\bm\psi$ a steady state of (\ref{eq:system}), it holds
$$0 = \del_x^2 \bm\psi + \bm f(\bm\psi), \qquad \del_x \bm\psi(0) = \del_x\bm\psi(1) = 0.$$
By the look of things, an analytical derivation of all steady-states of this system seems a little bit too optimistic. That being said, one can look at \textit{constant steady-states} $\bar{\bm u} = (\bar u, \bar v, \bar w)$ with $\bar u, \bar v, \bar w \in \R$. Proceeding as such leads to the term $\del_x^2 \bar{\bm u}$ vanishing, the problem reformulates into a less daunting one: finding a triplet of reals $\bar{\bm u}$ such that
$$\bm f(\bar{\bm u}) = 0.$$

Since we proved in the last section that solutions with positive initial values will stay positive for all time $t>0$, we are only interested in \textit{positive} equlibria of the system. Depending on the choice of parameters, one shows that there is either one (stable), two (stable-unstable) or three (stable-unstable-stable) such steady-states. In each scenario, the origin $(0, 0, 0)$ is a stable equilibrium point . For a specific range of parameters, another stable, positive equilibrium arises whose value turns out to be one of the roots of some third-order polynomial. To build geometrical intuition, the reader can see constant steady states as the intersection points of 3 hypersurfaces (see \ref{fig:nullclines}) algebraically described by the reaction term, i.e., 
$$\text{Steady states} = \left\{\bm x \in \R^3 : \bigcap_{\kappa = f, g, h} \bigl\{ \kappa(\bm x) = 0 \bigr\}\right\}.$$
\begin{figure}
	\begin{tabular}{cc}
		\label{fig:nullclines}
		\includegraphics[width=0.49\linewidth]{figures/nullclines.png}
		&
		\includegraphics[width=0.49\linewidth]{figures/nullclines_zoom.png}
	\end{tabular}
\caption{Three-dimensional plot of the algebraic hypersurfaces associated to equations $f \equiv 0$ (blue), $g \equiv 0$ (green) and $h \equiv 0$ (red). The choice of parameters $\mu_f = 0.87, \mu_b = 0.68, \mu_l = 0.05, \mu_e = 0.60, m_1 = 5.36, m2 = 9.68, m3 = 17.27$ is made so that three steady-states exist. The figure on the left-hand panel is plotted over the domain $[-10, 10]^3$ while the right-hand panel plotted in a neighborhood of the non-zero stable steady-state.}
\end{figure}

\subsubsection{Analytical expression of constant steady-states}

The strategy here is to first find, if any, functions $\euH$ and $\euG$ such that $\bar u = \euH(\bar v)$ and $\bar w = \euG(\bar v)$ and then derive a condition on $\bar v$ to conclude. We formulate

%\begin{proposition}[Constant stationary solutions]
%	Let $\bar{\bm u} = (\bar u, \bar v, \bar w)$ be a constant steady-state of (\ref{eq:redu})-(\ref{eq:redw}). Then, either $ (\bar u, \bar v, \bar w) = (0, 0, 0)$ or $\bar u = \euH(\bar v), \bar w = \euG(\bar v)$
%	and all possible values of $\bar v$ are given by the roots of $\euP$ where
%	$$\euH(x) = \frac{m_1}{\mu_f + \mu_b x} - \frac{1}{x}, \qquad \euG(x) = \frac{m_3}{\mu_e}\left( 1 - \frac{\mu_f+ \mu_b }{m_1 x} \right), \qquad \euP(x) = ax^3 + bx^2 + cx + d$$
%	and the coefficients of $\euP$ are given by
%	\begin{align*}
%		a &= -\ml \mb - \frac{m_3 \mb}{\me} + \frac{m_3 \mb^2}{m_1 \me}, \\[0.5em]
%		b &= -\ml\mf + m_2\mb - \frac{m_2 \mb^2}{m_1} +\mb^2 - m_1\mb - \frac{m_3 \mf}{\me} + \frac{2 \mf\mb m_3}{\me m_1}, \\[0.5em]
%		c &= m_2\mf - \frac{2\mf\mb m_2}{m_1} + \mf\mb + \frac{m_3 \mf^2}{\me m_1}, \\[0.5em] 
%		d &= -\frac{m_2\mf^2}{m_1}.
%	\end{align*}
%\end{proposition}


\begin{proposition}[Constant stationary solutions]
	Let $\bar{\bm u} = (\bar u, \bar v, \bar w)$ be a constant steady-state of (\ref{eq:redu})-(\ref{eq:redw}). Then, either $ (\bar u, \bar v, \bar w) = (0, 0, 0)$ or $\bar u = \euH(\bar v), \bar w = \euG(\bar v)$
	and all possible values of $\bar v$ are given by the roots of $\euP$ where
	$$\euH(x) = \frac{m_1}{\mu_f + \mu_b x} - \frac{1}{x}, \qquad \euG(x) = \frac{m_3}{\mu_e}\left( 1 - \frac{\mu_f+ \mu_b }{m_1 x} \right)$$
	and  $\euP$ is given by
	
	\begin{multline}
		\euP{(x)} = \biggl(-\ml \mb - \frac{m_3 \mb}{\me} + \frac{m_3 \mb^2}{m_1 \me}\biggr)x^3 \\+ \biggl(-\ml\mf + m_2\mb - \frac{m_2 \mb^2}{m_1} +\mb^2 - m_1\mb - \frac{m_3 \mf}{\me} + \frac{2 \mf\mb m_3}{\me m_1}\biggr)x^2 \\ + \biggl(m_2\mf - \frac{2\mf\mb m_2}{m_1} + \mf\mb + \frac{m_3 \mf^2}{\me m_1}\biggr)x -\frac{m_2\mf^2}{m_1}.
	\end{multline}
\end{proposition}
\begin{proof}
	The proof consists of straightforward computations that are carried out in details in appendix [NOM APPENDIX].
\end{proof}

While this is quite a ludicrous task to achieve, one could derive the exact analytical expression for spatially homogeneous stationary solutions of (\ref{eq:redu})-(\ref{eq:redw}) using the cubic formula and by differentiating cases with positive (negative) discriminant to toss away imaginary roots. An important thing to notice is that some choices of parameters will lead $\euP$ to have roots $v$ such that one coordinate of $\bar{\bm u}$ is negative, or complex-valued. Since we always choose an initial condition $\bm u_0 > 0$, this is contradictory with the results proven in the last section. When this happens, it is an indicator that the system will converge towards the origin instead of another steady-state. \com{maybe try to find parameters such that only one root is positive and two are complex-conjugate, e.g. $(r_1, r_2, r_3) = (3, 1+i, 1-i)$ such that $\euH(r_1), \euG(r_1) > 0$ to see if it is possible to get another steady state that is not the origin.}

\begin{example}[Numerical computations of the roots]
	In MATLAB, we implement the polynomial $\euP$ as well as $\euG$ and $\euH$. We then use the function \texttt{roots} to find all eventual steady states (see figure (\ref{fig:matlabP})).
	\begin{figure}
		\label{fig:matlabP}
		\includegraphics[width=0.6\linewidth]{figures/polyP_v.png}
		\caption{Roots of $\euP$ for $\mu_f = 0.87$, $\mu_b = 0.68$, $\mu_l = 0.05$, $\mu_e = 0.60$, $m_1 = 5.36$, $m2 = 9.68$, $m3 = 17.27$ (same parameters as in figure (\ref{fig:nullclines})) }
	\end{figure}
	Using the exact values returned by the machine, we find four steady-states 
	$$\bar{\bm u}_{3} = (0, 0, 0), \quad \bar{\bm u}_{2} = (-51.6587, -1.4300, 28.3989) $$
	$$\bar{\bm u}_{3} = (1.6527, 0.2994, 9.5284), \quad \bar{\bm u}_{4} = (0.0138, 0.1864, 0.0739)$$
	As previously mentioned, we can rule out $\bar{\bm u}_2$ because it has at least one negative coordinate. From there, using the Jacobian (computed in the next section) we can deduce the nature of $\bar{\bm u}_1, \bar{\bm u}_3, \bar{\bm u}_4$ and it reveals that the origin and $\bar{\bm u}_3$ are stable while $\bar{\bm u}_4$ is unstable.
\end{example}
	
	

\subsubsection{Snack break: finding bounds on asymptotically stable spatially homogeneous stationary solutions}

hey hey


\subsection{Linear dynamics and local behavior of the system}

In order to meet the required conditions for DDI, we first have to make sure system (\ref{eq:redu})-(\ref{eq:redw}) is stable in absence of diffusion when computed at a spatially homogeneous steady-state. This translates into saying that the Jacobian matrix $J = J_{\bm f}(u, v, w)$ of $\bm f$ at a steady-state satisfies $s(J) < 0$. A rapid computation shows

$$
J = \bpm  -\mu_f +  \dfrac{m_1v}{(1+u v)^2} - \mu_b v &  \dfrac{m_1 u}{(1+u v)^2} - \mu_b u & 0 \\[1.2em]
 \dfrac{m_2 v}{(1+u v)^2} - \mu_b v   & -\mu_l + \dfrac{m_2 u}{(1+u v)^2} - \mu_b u - w  & -v \\[1.2em] 
  \dfrac{m_3 v}{(1+u v)^2} &  \dfrac{m_3 v}{(1+u v)^2} & -\mu_e \epm.
$$

When working around the origin, it is perfectly fine to work with $J$ as it is. However, this form turns out to be rather impractical to work with when carrying out later computations around the other non-zero steady state. As a parry, we will later derive a set of useful equalities that will help us simplify the expression of $J$.

\subsubsection{In a neighborhood of the origin}

When taking $(u, v, w) = (0, 0, 0)$, almost all terms in the Jacobian disappear, leaving us with

$$
A(0, 0, 0) = \bpm  -\mu_f  &  0 & 0 \\[1.2em]
0  & -\mu_l & 0 \\[1.2em] 
0 & 0 & -\mu_e \epm
$$

which, by positivity of $\mu_f, \mu_b, \mu_e$ is unconditionally stable. Take a moment to convince yourself that in this case, there is no hope for DDI (to see why, notice that $\tr (A-\bm\mu D) < 0$ and $\det (A-\bm\mu D) = -\mu_f(\mu_l+ \bm\mu / \gamma)(\mu_e + \bm\mu d_2 / \gamma) < 0$ for all non-negative choices of $\bm\mu$). This tells us the system will always be stable, a statement that is corroborated by running a few simulations (see figure \ref{fig:simulations}).

\begin{figure}
	\centering
	\includegraphics[width=\textwidth]{figures/stable_origin.png}
	\caption{space-time plot of numerical solutions $u, v, w$ to \ref{eq:system} with constant initial condition $\bm X_0 = (\uu_0, \vv_0, \ww_0)$ close to the origin. One clearly sees how each quantity converges towards 0}
	\label{fig:simulations}
\end{figure}

\subsubsection{Around the other steady state}

Let us move our interest to the other stable steady state (provided the choice of parameters allows it to exist). Using the fact that each term $u, v, w$ is positive, we can operate some surgery on identities \ref{identities} to derive

\begin{equation}
	\label{new_identities}
		m_1 \dfrac{v}{1 + u v} = \mu_f +  \mu_b v \quad \quad
		m_2 \dfrac{u}{1 + u v}  = \mu_l +  \mu_b u + w \quad \quad
		m_3 \dfrac{ u v}{1 + uv} = \mu_e w
\end{equation}


\begin{theorem}[Routh-Hurwitz criterion of order 3]
	Take a matrix $M \in \euM_3(\R)$. Then all the roots of $\chi_M$ lie in the negative half-plane (i.e. $\sigma(M) \subset \R_{<0}$) if and only if
	$\Delta_i(M) > 0$ for $i=1, 2, 3$ holds, where we define
	\begin{align*}
		\Delta_1(M) &= -\tr(M) \\[0.7em]
		\Delta_2(M) &= -\tr(M) \sum_{i < j} \det(M_{ij}) + \det(M) \\[0.7em]
		\Delta_3(M) &= -\det(M) \Delta_2(M),
	\end{align*}
Alternatively, let $\Tilde{\chi}_{M}(\lambda) = a_0 + a_1 \lambda + a_2 \lambda^2 + \lambda^3$ be the normalized characteristic polynomial of $M$. Then all roots of $\Tilde{\chi}_M$ lie in the negative half-plane if and only if all coefficients are positive and $a_2 a_1 > a_0$.
\end{theorem}


\begin{definition}[Submatrices defined by a pair of indices]
	Consider a matrix $A \in \euM_n(\R)$, for any couple of distinct indices $(i, j)$ with $1 \le i, j \le n$, we introduce the notation $A_{ij}$ to be the $2 \! \times \! 2$-matrix whose entries are formed by the entries of $A$ on line and column $i, j$.
\end{definition}


\begin{proposition}[Characteristic polynomial of a $3\times3$ matrix] Consider a matrix $A \in \euM_3(\R)$ whose entries are the $a_{ij}$ for $1 \le i, j \le 3$. Then 
	$$\chi_A(\lambda) = -\lambda^3 + \tr(A) \lambda^2 - \lp \sum_{i < j} \det(A_{ij})\rp \lambda + \det(A)$$.
\end{proposition}


\begin{lemma}[Necessary condition for DDI] Let $B := A - \lambda D$ denote the matrix of the linearized system. We claim that we obtain DDI if and only if $\det(B_{12}) < 0$ and 
	
\end{lemma}


\begin{remark}[Minimum] The function $\mu \longmapsto |A - \mu D|$ reaches its minimum at the point:
	
\begin{equation}
	\resizebox{\hsize}{!}{%
		$\mu_{\mathbf{\mathrm{min}}} = \dfrac{\gamma}{2}\left( \dfrac{1}{(1+uv)^2}\left( m_2 u -  \dfrac{m_1m_2uv + \mu_b^2 uv(1+uv)^4 - \mu_b(m_1 + m_2) uv(1+uv)^2}{m_1 v - \mu_f (1 + uv)^2 - \mu_b v(1+uv)^2} \right)-\dfrac{\mu_e}{d_2} - \mu_l - \mu_b u - w\right) \notag$  
	}
\end{equation}
Please do not make me compute $\det(A - \mu_{\textrm{min}} D)$ ;-;. Also
\begin{equation}
	\resizebox{\hsize}{!}{%
		$\mu_{\mathbf{\mathrm{min}}} > 0 \quad \iff\quad \mu_f \biggl(m_2 - \mu_b u (1 + uv)^2\biggr) > \left( \dfrac{\mu_e}{d_2} + \mu_l + w \right) \biggl( m_1 v - \mu_f(1+uv)^2 - \mu_b v (1 + uv)^2\biggr) \notag$  
	}
\end{equation}
\end{remark}



$$\frac{|A_{12}| + |A_{13}|d_2}{2a_{11} d_2}$$


