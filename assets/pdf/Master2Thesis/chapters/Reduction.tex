\section{Reduction of the Model}

In order to simplify the system, we use a two-step approach. First, by applying a quasi-steady-state approximation, and then by using a change of variable to further reduce the amount of parameters. 

\subsection{Quasi-steady-state approximation}

The Quasi-Steady-State Approximation (QSSA) is a  technique inspired from the field of  chemical kinetics or more generally biochemistry. When introduced, the purpose of such an approximation is to simplify the analysis by assuming that certain chemical species are reaching their steady-state concentrations quicker than other species in the system.

Previously used in an \textit{ad hoc} fashion by biologists, they theory behind QSSA has been thoroughly explored and is now carefully described thanks to the framework provided by singular perturbation theory. In particular for equations emerging from chemistry.
When performing QSSA, the rate of change of concentrations of these slower species is assumed to be negligible compared to the rates of other reactions in the system. Therefore, the concentrations of these species can be approximated as constants during the time course of the reaction.

\begin{remark}
Although the QSSA can be proven to be physically relevant when applied to some systems (Reaction-Diffusion equations are a good example), the approximation may not always hold true under certain conditions.
\end{remark}

Coming back to our system, we perform a QSSA on the quantity $r_b$, \textit{i.e.}, we assume $\delt{r_b} \equiv 0$. and find that $(d + \mu_b)\bm{r_b} = b \uu \vv $, therefore

$$\bm{r_b} = \dfrac{b}{d + \mu_b} \uu \vv$$
For what follows, we define $\alpha = b / (d + \mu_b)$ and replace the newly found value of $\bm{r_b}$ in the system.

\begin{align}
    \dfrac{\del}{\del_t} \uu &= -\mu_f \uu + m_1 \dfrac{\alpha \uu \vv}{1 + \alpha \uu \vv} - b \uu \vv + d \alpha \uu \vv \notag \\[1em]
    \dfrac{\del}{\del_t} \vv &= \dfrac{1}{\gamma} \delxx{\vv} - \mu_l \vv + m_2 \dfrac{\alpha \uu \vv}{1 + \alpha \uu\vv} - b \uu \vv + d \alpha \uu \vv - b_e \vv\ww  \\[1em]
    \dfrac{\del}{\del_t} \ww &= \dfrac{d_2}{\gamma} \delxx{\ww} - \mu_e \ww + m_3 \dfrac{\alpha\uu\vv}{1 + \alpha\uu\vv} \notag
\end{align}

\subsection{Change of variable}

Now that we have reduced the amount of variables from four to three, let us also operate surgery on the system to get rid of some parameters. First notice that $d\alpha - b = -\mu_b \alpha$, and then proceed to the change of variable 

\begin{center}
\begin{tabular}{cccccccc}
	new variable / parameter & $\Tilde{\uu}$ & $ \Tilde{\vv}$ & $ \Tilde{\ww}$ & $\Tilde{m}_1 $ & $\Tilde{m}_2$ & $\Tilde{m}_3$ & $\Tilde{\mu_b}$  \\[0.8em]
	value & $\sqrt{\alpha} \uu$ & $\sqrt{\alpha} \vv$ & $b_e \ww$ & $\sqrt{\alpha} m_1$ & $\sqrt{\alpha} m_2$ & $\sqrt{\alpha} m_3$ &  $\sqrt{\alpha} \mu_b $
\end{tabular}
\end{center}

the new system is then simpliying down to

\begin{align}
	\label{system}
	 \dfrac{\del}{\del_t} \Tilde{\uu} &= -\mu_f \Tilde{\uu} + \Tilde{m}_1 \dfrac{ \Tilde{\uu} \Tilde{\vv}}{1 +  \Tilde{\uu} \Tilde{\vv}} - \Tilde{\mu_b} \Tilde{\uu} \Tilde{ \vv}  \notag \\[1em]
	\dfrac{\del}{\del_t} \Tilde{\vv} &= \dfrac{1}{\gamma} \delxx{\Tilde{\vv}} - \mu_l  \Tilde{\vv} +\Tilde{ m}_2 \dfrac{ \Tilde{\uu} \Tilde{\vv}}{1 +  \Tilde{\uu}\Tilde{\vv}} - \Tilde{\mu_b} \Tilde{ \uu} \Tilde{ \vv} - \Tilde{\vv} \Tilde{\ww}  \\[1em]
	\dfrac{\del}{\del_t} \Tilde{\ww} &= \dfrac{d_2}{\gamma} \delxx{\Tilde{\ww}} - \mu_e\Tilde{ \ww} + \Tilde{m}_3 \dfrac{\Tilde{\uu}\Tilde{\vv}}{1 + \Tilde{\uu}\Tilde{\vv}} \notag
\end{align}

For the sake of readability, we drop the $\sim$ notation in the future.

\subsection{Invariant Region}

In this section, we prove the existence of a region $\Sigma$ such that, whenever the initial condition lies in $\Sigma$, we have existence of solutions for all $t > 0$. For that we consider the function $f = (f^1, f^2, f^3)$ such that $\del_t (u, v, w) = f(u, v, w)$.

\begin{lemma}[boundedness of solutions]
	There exists three positive reals $A_u, A_v, A_w$ such that the region $$\Sigma = \bigl\{(u, v, w) : 0 \le u \le A_u, \quad 0 \le v \le A_v, \quad  0 \le w \le A_w, \bigr\}$$ is $f$-invariant, i.e. the vector field $\phi_f$ generated by $f$ never points outwards of $\Sigma$
\end{lemma}

\begin{proof}
	We proceed in two steps. First we show that solutions with a non-negative initial condition will always stay non-negative for all $t>0$, which is significant from a  biological modeling point of view. Then we prove that there exists an upper bound for each quantity (no explosion in finite time). Let us start by rewriting $$\Sigma = \Sigma_0 \cap \Sigma_A =: \bigcap_i \biggl( \{G_i \le 0\} \cap \{H_i \le 0\}\biggr), \qquad i \in \{u, v, w\}$$
	The rectangular region (which is clearly convex) where each $G_i$ and $H_i$ are smooth functions prescribing the constraints on $\Sigma$. They are defined as follows
	
	\begin{align*}
		& G_u(u, v, w) = -u & H_u(u, v, w) = u - A_u \\[1em]
		& G_v(u, v, w) = -v & H_v(u, v, w) = v - A_v \\[1em]
		& G_w(u, v, w) = -w & H_w(u, v, w) = w - A_w
	\end{align*}

	Then, we define $\del\Sigma := \{X \in \Sigma : \exists i, \ \  G_i(X) = 0 \lor H_i(X) = 0 \} = \del \Sigma_0 \cap \del \Sigma_A$. Let us take $X \in \del \Sigma_0$. If $u=0$, then $$(\grad G_u \cdot \phi_f)(X)\bigl\vert_{u=0} = u \left( \mu_f  + \mu_b v - m_1 \dfrac{v}{1 + uv}\right)\biggl\vert_{u=0} = 0.$$ The case $v= 0$ is similar in a way that
	$$(\grad G_v \cdot \phi_f)(X)\bigl\vert_{v=0} = v \left( \mu_l  + \mu_b u + w - m_2 \dfrac{u}{1 + uv}\right)\biggl\vert_{v=0} = 0.$$ Finally, since $X \in \del\Sigma_0$, it holds that $u, v \ge 0$, which means that if $w=0$, then $$(\grad G_w \cdot \phi_f)(X)\bigl\vert_{w=0} = -m_3 \dfrac{uv}{1 + uv} < 0$$ This proves that $\Sigma_0$ is invariant by $f$. Let us proceed in similar fashion to prove that $\Sigma_A$ is also $f$-invariant. First, we notice that
	
	\begin{align}
		f^1(u, v, w) &= -\mu_f + m_1 \dfrac{uv}{1 + uv} - \mu_b uv \notag \\[0.5em]
		& \le m_1 - \min(\mu_f, \mu_b) u (1 + v) \notag
	\end{align}
	
	Using the fact that $X \in \Sigma$, we deduce $1 + v \ge 1$ and therefore we can get rid of this term in the product, leaving us with 
	$$f^1(u, v, w) \le m_1 - \min(\mu_f, \mu_b) u.$$
	In other words, we can find a real $A_u \in \R_{>0}$ such that $A_u > m_1 / \min(\mu_f, \mu_b)$. This implies $(\grad H_u \cdot \phi_f)(X)\vert_{u=A_u} = f^1(A_u, v, w) \le 0$. The same logic applies to show $$f^2(u, v, w) \le m_2 - \min(\mu_f, \mu_b, 1) v.$$ Thus, by picking $A_v \ge m_2 / \min(\mu_f, \mu_b, 1)$, we get $(\grad H_v \cdot \phi_f)(X)\vert_{v=A_v} \le 0$. To conclude this proof, we show that $$f^3(u, v, w) \le m_3 - \mu_e w,$$ and take $A_w \ge m_3 / \mu_e$ to obtain $(\grad H_w \cdot \phi_f)(X)\vert_{w=A_w} \le 0$. Theorem (3.16) from [AMC] strikes the final blow.	
\end{proof}

This powerful theorem enables us to guarantee the existence of solutions to \ref{system} coupled with homogeneous Neumann boundary conditions under biologically relevant initial conditions. 


