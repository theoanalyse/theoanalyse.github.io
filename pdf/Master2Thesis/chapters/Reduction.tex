\section{Reduction of the Model}

In order to simplify the system, we use a two-step approach. First, by applying a quasi-steady-state approximation, and then by using a change of variable to further reduce the amount of parameters. 

\subsection{Quasi-steady-state approximation}

The Quasi-steady-state (QSS) approximation is a  technique inspired from the field of  chemical kinetics or more generally biochemistry. When introduced, the purpose of such an approximation is to simplify the analysis by assuming that certain chemical species are reaching their steady-state concentrations quicker than other species in the system.

Previously used in an \textit{ad hoc} fashion by biologists, QSS Approximation is now carefully described thanks to the framework provided by singular perturbation theory for equations emerging from chemistry.
When performing QSS Approximation, the rate of change of concentrations of these slower species is assumed to be negligible compared to the rates of other reactions in the system. Therefore, the concentrations of these species can be approximated as constants during the time course of the reaction.

\begin{remark}
Although the QSS Approximation can be proven to be physically relevant when applied to some systems (Reaction-Diffusion equations are a good example), the approximation may not always hold true under certain conditions.
\end{remark}

Coming back to our system, we perform a QSSA on the quantity $r_b$, \textit{i.e.}, we assume $\delt{r_b} \equiv 0$. and find that $(d + \mu_b)\bm{r_b} = b \uu \vv $, therefore

$$\bm{r_b} = \dfrac{b}{d + \mu_b} \uu \vv$$
For what follows, we define $\alpha = b / (d + \mu_b)$ and replace $\bm{r_b}$ in the system.

\begin{align}
    \dfrac{\del}{\del_t} \uu &= -\mu_f \uu + m_1 \dfrac{\alpha \uu \vv}{1 + \alpha \uu \vv} - b \uu \vv + d \alpha \uu \vv \notag \\[1em]
    \dfrac{\del}{\del_t} \vv &= \dfrac{1}{\gamma} \delxx{\vv} - \mu_l \vv + m_2 \dfrac{\alpha \uu \vv}{1 + \alpha \uu\vv} - b \uu \vv + d \alpha \uu \vv - b_e \vv\ww  \\[1em]
    \dfrac{\del}{\del_t} \ww &= \dfrac{d_2}{\gamma} \delxx{\ww} - \mu_e \ww + m_3 \dfrac{\alpha\uu\vv}{1 + \alpha\uu\vv} \notag
\end{align}

\subsection{Change of variable}

Now that we have reduced the amount of variables from four to three, let us also operate surgery on the system to get rid of some parameters. First notice that $d\alpha - b = -\mu_b \alpha$, and then proceed to the change of variable 

\begin{center}
\begin{tabular}{cccccccc}
	new variable / parameter & $\Tilde{\uu}$ & $ \Tilde{\vv}$ & $ \Tilde{\ww}$ & $\Tilde{m}_1 $ & $\Tilde{m}_2$ & $\Tilde{m}_3$ & $\Tilde{\mu_b}$  \\[0.8em]
	value & $\sqrt{\alpha} \uu$ & $\sqrt{\alpha} \vv$ & $b_e \ww$ & $\sqrt{\alpha} m_1$ & $\sqrt{\alpha} m_2$ & $\sqrt{\alpha} m_3$ &  $\sqrt{\alpha} \mu_b $
\end{tabular}
\end{center}

the new system is then simpliying down to

\begin{align}
	 \dfrac{\del}{\del_t} \Tilde{\uu} &= -\mu_f \Tilde{\uu} + \Tilde{m}_1 \dfrac{ \Tilde{\uu} \Tilde{\vv}}{1 +  \Tilde{\uu} \Tilde{\vv}} - \Tilde{\mu_b} \Tilde{\uu} \Tilde{ \vv}  \notag \\[1em]
	\dfrac{\del}{\del_t} \Tilde{\vv} &= \dfrac{1}{\gamma} \delxx{\Tilde{\vv}} - \mu_l  \Tilde{\vv} +\Tilde{ m}_2 \dfrac{ \Tilde{\uu} \Tilde{\vv}}{1 +  \Tilde{\uu}\Tilde{\vv}} - \Tilde{\mu_b} \Tilde{ \uu} \Tilde{ \vv} - \Tilde{\vv} \Tilde{\ww}  \\[1em]
	\dfrac{\del}{\del_t} \Tilde{\ww} &= \dfrac{d_2}{\gamma} \delxx{\Tilde{\ww}} - \mu_e\Tilde{ \ww} + \Tilde{m}_3 \dfrac{\Tilde{\uu}\Tilde{\vv}}{1 + \Tilde{\uu}\Tilde{\vv}} \notag
\end{align}

For the sake of readability, we drop the $\sim$ notation in the future.



